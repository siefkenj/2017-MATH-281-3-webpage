\documentclass[letter]{article}
\usepackage{amsmath}
\usepackage{amsfonts}
\usepackage{amssymb}
\usepackage{ifthen}
\usepackage{fancyhdr}
\usepackage{enumitem}

%%%
% Set up the margins to use a fairly large area of the page
%%%
\oddsidemargin=.2in
\evensidemargin=.2in
\textwidth=6in
\topmargin=0in
\textheight=9.0in
\parskip=.07in
\parindent=0in
\pagestyle{fancy}

%%%
% Set up the header
%%%
\newcommand{\setheader}[6]{
	\lhead{{\sc #1}\\{\sc #2} ({\small \it \today})}
	\rhead{
		{\bf #3} 
		\ifthenelse{\equal{#4}{}}{}{(#4)}\\
		{\bf #5} 
		\ifthenelse{\equal{#6}{}}{}{(#6)}%
	}
}

%%%
% Set up some shortcut commands
%%%
\newcommand{\R}{\mathbb{R}}
\newcommand{\C}{\mathbb{C}}
\newcommand{\N}{\mathbb{N}}
\newcommand{\Z}{\mathbb{Z}}
\newcommand{\Proj}{\mathrm{proj}}
\newcommand{\Perp}{\mathrm{perp}}
\newcommand{\proj}{\mathrm{proj}}
\newcommand{\Span}{\mathrm{span}}
\newcommand{\Null}{\mathrm{null}}
\newcommand{\Rank}{\mathrm{rank}}
\newcommand{\mat}[1]{\begin{bmatrix}#1\end{bmatrix}}
\renewcommand{\d}{\mathrm{d}}

%%%
% This is where the body of the document goes
%%%
\begin{document}
	\setheader{Math 281-3}{Homework 2}{Due Thursday, April 13}{}{}{}
	\begin{enumerate}
		\item Is the set $X=\left\{\mat{1\\2\\3},\mat{4\\5\\6},\mat{7\\8\\9}\right\}$ a basis for
			$\R^3$?  Justify your answer using technical linear algebra vocabulary (including definitions).
		\item Recall the set $\mathcal G$ of all $4\times 4$ grayscale images from Lab 1.
			Is $\mathcal G$ a subspace?  If so, find a basis for it and its dimension.
		\item \begin{enumerate}
				\item Give an example of a $3\times 3$ non-homogeneous system of equations
					that (i) has one solution, (ii) has infinitely many solutions, (iii)
					has no solutions.  Make sure to explain why you know each system has the desired property.
				\item Can you give an example of a $3\times 3$ homogeneous system
					for properties (i), (ii), and (iii)?  Explain why or why not.
			\end{enumerate}
		\item \begin{enumerate}
				\item Use an augmented matrix to solve
					\begin{align*}
						x+y&=7\\
						2x-3y&=13.
					\end{align*}
					Are there any values you could replace the right hand side of the equations with
					such that there would be no solution?  Explain using technical linear algebra terms like
					basis, span, linear independence/dependence, subspace, etc.  (Yes, this probablem should
					look just like the one from Homework 1!)
				\item 
				Consider the system given by the augmented matrix
				\[
					C=\left[\begin{array}{ccccc|c}
						1&0&1&2&0&-1\\
						0&1&1&0&0&3\\
						0&0&0&0&1&4
					\end{array}\right].
				\]
				and call the variables in this system $x_1,x_2,
				x_3,x_4,x_5$.  Write all solutions to this system in vector form.
				\item There are 10 ways to pick two things from the set $\{x_1,x_2,x_3,x_4,x_5\}$.
					For each of the ten ways, determine whether that pair is a valid choice
					of free variables for $C$.
				\item Write down all solutions to the homogeneous system corresponding to $C$.  How does this
					set of solutions correspond to the solutions to $C$?
				\item Let $H$ be the homogeneous system corresponding to $C$.
					For every choice of free variables you make when solving $H$, you get solutions of
					the form
					\[
						\vec x = t\vec d_1+s\vec d_2.
					\]
					Let $\Gamma=\{\vec d_1,\vec d_2\}$ be the set of
					vectors $\vec d_1$ and $\vec d_2$ corresponding to one choice
					of free variables and let $\Delta=\{\vec d_1,\vec d_2\}$ be the set
					of vectors $\vec d_1$, $\vec d_2$ arising from a \emph{different} choice
					of free variables.

					Will the set $\Gamma\cup \Delta$ be linearly independent or dependent?
					Does this depend on your choice?  Give a basis for $\Span(\Gamma\cup\Delta)$.
					(Hint: if this is hard to think about, work out some examples with some actual
					choices.)
			\end{enumerate}
		\item Let $X$ be an unknown $3\times 3$ matrix.  List every possible reduced row echelon form
			for $X$.  You may use the $?$ or $*$ symbols as place holders for ``any number,'' but
			if a certain position in the matrix must be filled with a particular number, put that number.
			For each reduced row echelon form, identity how many free variables there are and the rank
			of $X$.  (Hint: there
			are a fair number of possible reduced row echelon forms for a $3\times 3$).
		\item Let $X$ be an unknown $3\times 3$ system and let $S=\{\text{solutions to }X\}$.
			Show that $S$ is a subspace if and only if $X$ is a homogeneous system.

	\end{enumerate}

\end{document}
