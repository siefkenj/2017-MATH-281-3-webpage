\documentclass[letter]{article}
\usepackage{amsmath}
\usepackage{amsfonts}
\usepackage{amssymb}
\usepackage{ifthen}
\usepackage{fancyhdr}
\usepackage{enumitem}
\usepackage[hidelinks]{hyperref}
\usepackage{tikz}

%%%
% Set up the margins to use a fairly large area of the page
%%%
\oddsidemargin=.2in
\evensidemargin=.2in
\textwidth=6in
\topmargin=-.4in
\textheight=9.0in
\parskip=.07in
\parindent=0in
\pagestyle{fancy}

%%%
% Set up the header
%%%
\newcommand{\setheader}[6]{
	\lhead{{\sc #1}\\{\sc #2} ({\small \it \today})}
	\rhead{
		{\bf #3} 
		\ifthenelse{\equal{#4}{}}{}{(#4)}\\
		{\bf #5} 
		\ifthenelse{\equal{#6}{}}{}{(#6)}%
	}
}

%%%
% Set up some shortcut commands
%%%
\newcommand{\R}{\mathbb{R}}
\renewcommand{\C}{\mathbb{C}}
\newcommand{\N}{\mathbb{N}}
\newcommand{\Z}{\mathbb{Z}}
\newcommand{\Proj}{\mathrm{proj}}
\newcommand{\Perp}{\mathrm{perp}}
\newcommand{\proj}{\mathrm{proj}}
\newcommand{\Span}{\mathrm{span}}
\newcommand{\Null}{\mathrm{null}}
\newcommand{\Rank}{\mathrm{rank}}
\newcommand{\mat}[1]{\begin{bmatrix}#1\end{bmatrix}}
\renewcommand{\d}{\mathrm{d}}

%%%
% This is where the body of the document goes
%%%
\begin{document}
	\setheader{Math 281-3}{Homework 4}{Due Thursday, April 27}{}{}{}
	\begin{enumerate}
		\item For each of the following statements, produce a counterexample to show that the statement is {\bf false}.
			\begin{enumerate}
				\item If $A$ and $B$ are square matrices, $AB=BA$.
				\item If $AB=\mat{1&1\\1&1}$, then $A$ and $B$ are $2\times 2$ matrices.
				\item If $AB=I$ then $BA=I$.
				\item If $A^2=0$, then $A=0$.
			\end{enumerate}
		\item Let $R=\mat{1&2&3\\4&5&6\\7&8&9}$.
			\begin{enumerate}
				\item Find all solutions to the matrix equation $R\mat{x_1\\x_2\\x_3}=\mat{2\\5\\8}$.
				\item Prove that the set $X=\{\vec x\in\R^3:R\vec x=\vec 0\}$ is a subspace.
				\item Prove that the set $Y=\{\vec y\in\R^3:\vec y=R\vec x\text{ for some }\vec x\in\R^3\}$ is a subspace.
			\end{enumerate}
		\item This problem explores two interpretations of matrix multiplication: \emph{linear combinations of
			columns} and \emph{dot products with rows}.  If you need ideas, please review these two
			interpretations.
			
			Suppose $E$ is a $4\times 3$ matrix with columns $\vec c_1,\vec c_2,\vec c_3$ and rows
			$\vec r_1,\vec r_2,\vec r_3,\vec r_4$.  Let $\vec v=\mat{2\\-1\\1}$.
			\begin{enumerate}
				\item Express $E\vec v$ as a linear combination of $\vec c_1,\vec c_2,\vec c_3$.
				\item Supposing $\vec r_1\cdot \vec v=1$, $\vec r_2\cdot \vec v=6$, $(\vec r_3+\vec r_4)\cdot \vec v=2$,
					and $(\vec r_3-\vec r_4)\cdot \vec v=-2$, compute $E\vec v$.
			\end{enumerate}

		\item Suppose that $\vec u$, $\vec v$, and $\vec w$ are vectors in $\R^2$ that are related
			by the following diagram of a parallelogram.
		\begin{center}
			\begin{tikzpicture}[scale=.5]
				\draw[-] (-6, 0) -- node [below,
				very near end] {} (6, 0);
				\draw[-] (0, -2) -- node [right,
				very near start] {} (0, 5);
				\draw[->] (0, 0) -- node [below,
				very near end] {$4\vec{u}$}
				(3, -1);
				\draw[->] (0, 0) -- node [left,
				near end] {$2\vec{v}$} (2, 3);
				\draw[->] (0, 0) -- node [left,
				very near end] {$3\vec{w}$}
				(-1, 4);
				\draw[dotted] (3, -1) -- node
				[right, very near end] {} (2, 3);
				\draw[dotted] (-1,4) -- node
				[below, very near end] {} (2, 3);
			\end{tikzpicture}
		\end{center}
		Let $A=[\vec u|\vec v|\vec w]$ be the matrix with columns $\vec u$, $\vec v$, and $\vec w$.
		\begin{enumerate}
			\item What is the rank of $A$?
			\item Find all solutions to the equation $A\vec x=\vec 0$.
			\item Find a basis for the subspace $V=\{\vec x\in\R^3: A\vec x=\vec 0\}$.
		\end{enumerate}



		\item 
		\begin{enumerate}
		
			\item For each of the following linear transformations, find a matrix corresponding to the transformation.
		 \begin{enumerate}
				\item $\mathcal{A}$: Rotate $90^\circ$ counterclockwise around the origin (in $\R^2$).
				\item $\mathcal{B}$: Send every vector in $\R^2$ to the zero vector.
				\item $\mathcal{C}$: Project (in $\R^3$) onto the $yz$-plane.
				\item $\mathcal{D}$: Project (in $\R^3$) onto the $x$-axis.
				\item $\mathcal{E}$: Reflect (in $\R^3$) across the $xy$-plane.
				\item $\mathcal{F}$: Stretch by a factor of 2 in the $y$-direction (in $\R^2$).
				\item $\mathcal{G}$: Stretch by a factor of 2 in the $y$-direction (in $\R^3$).
				\item $\mathcal{H}$: Rotate $90^\circ$ counterclockwise (as viewed looking ``down'' from the positive $y$-axis towards the origin) around the $y$-axis (in $\R^3$).
			\end{enumerate}
		
		
		\item For each transformation in part (a), explain geometrically whether or not the transformation is invertible.
			(Recall, a function $f$ is invertible if there exists another function $g$ so that $f\circ g$ \emph{and}
				$g\circ f$ are both the identity function.)
		
		
		\item Let
			\[
				X=\mat{0 & -1 \\ 2 & 0}
				\qquad \qquad
				Y=\mat{0 & 0 & 0\\ 0 & 1 & 0 \\ 0 & 0 & -1}
			\]
			\[
				Z=\mat{0 & 0 & 1 \\ 0 & 2 & 0 \\ -1 & 0 & 0}
				\qquad\qquad
				W=\mat{0 & 0 & 1 \\ 0 & 0 & 0 \\ 0 & 0 & 0}
			\]
			For each matrix $M\in\{X,Y,Z,W\}$, define a transformation $T_M$ where $T_M(\vec v) = M\vec v$.

			For each $M$, explain how to obtain $T_M$ as a combination of linear transformations from part (a).
				(Hint: you may need to combine two, three, or more transformations from part (a).)
		\end{enumerate}

		\item Let $\{\vec v_1,\vec v_2,\vec v_3\}\subset \R^3$ be a linearly independent set.
			\begin{enumerate}
				\item Suppose a linear transformation $T:\R^3\to\R^3$ satisfies $T(\vec v_i) = \vec v_i$ for $i\in\{1,2,3\}$.
					Prove that $T$ is the identity transformation.
				\item Suppose a linear transformation $S:\R^3\to\R^3$ satisfies $S(\vec v_1)=\vec v_1$,
					$S(\vec v_2)=\vec v_2$ and $S(\vec v_3)=5 v_3$.  Prove that $S$ is invertible.
			\end{enumerate}
	\end{enumerate}


\end{document}
