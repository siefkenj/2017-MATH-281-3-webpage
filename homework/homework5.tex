\documentclass[letter]{article}
\usepackage{amsmath}
\usepackage{amsfonts}
\usepackage{amssymb}
\usepackage{ifthen}
\usepackage{fancyhdr}
\usepackage{enumitem}
\usepackage[hidelinks]{hyperref}
\usepackage{tikz}

%%%
% Set up the margins to use a fairly large area of the page
%%%
\oddsidemargin=.2in
\evensidemargin=.2in
\textwidth=6in
\topmargin=-.4in
\textheight=9.0in
\parskip=.07in
\parindent=0in
\pagestyle{fancy}

%%%
% Set up the header
%%%
\newcommand{\setheader}[6]{
	\lhead{{\sc #1}\\{\sc #2}}
	\rhead{
		{\bf #3} 
		\ifthenelse{\equal{#4}{}}{}{(#4)}\\
		{\bf #5} 
		\ifthenelse{\equal{#6}{}}{}{(#6)}%
	}
}

%%%
% Set up some shortcut commands
%%%
\newcommand{\R}{\mathbb{R}}
\renewcommand{\C}{\mathbb{C}}
\newcommand{\N}{\mathbb{N}}
\newcommand{\Z}{\mathbb{Z}}
\newcommand{\Proj}{\mathrm{proj}}
\newcommand{\Perp}{\mathrm{perp}}
\newcommand{\proj}{\mathrm{proj}}
\newcommand{\Span}{\mathrm{span}}
\newcommand{\Null}{\mathrm{null}}
\newcommand{\Rank}{\mathrm{rank}}
\newcommand{\mat}[1]{\begin{bmatrix}#1\end{bmatrix}}
\renewcommand{\d}{\mathrm{d}}

%%%
% This is where the body of the document goes
%%%
\begin{document}
\setheader{Math 281-3}{Homework 5}{Due Thursday, May 11}{}{}{}

	\begin{enumerate}
		\item Suppose the matrix equation $A\vec x=\mat{3\\2\\7}$ has the general solution
			\[
				\vec x=\mat{1\\0\\0}+s\mat{1\\1\\0}+t\mat{-1\\0\\1}.
			\]
			\begin{enumerate}
				\item How many rows and how many columns does $A$ have?
				\item Find $\Null(A)$.
				\item Find $\Rank(A)$.
				\item Find $\text{col}(A)$.
				\item Find $\text{row}(A)$.
			\end{enumerate}
	\item {\sc Not all bases are created equally}.
		Let $\mathcal X=\{\vec x_1,\vec x_2\}$ be a basis for $\R^2$ consisting
			of unit vectors.  By definition, for any
			$\vec v\in\R^2$, we can find a unique $\alpha_1,\alpha_2\in\R^2$ so that
		\[
			\vec v=\alpha_1\vec x_1+\alpha_2\vec x_2.
		\]
		We call the quantity $S_{\mathcal X}(\vec v)=\sqrt{\alpha_1^2+\alpha_2^2}/\|\vec v\|$ 
		the \emph{size of the representation of of $\vec v$ in the basis $\mathcal X$}.
		For many applications, it is desirable for $S_{\mathcal X}$ to be as small
		as possible.  We are going to explore what makes $S_{\mathcal X}$ small
		or large.

		\begin{enumerate}
			\item Let 
				\[
					\vec a'=\mat{5\\5}\qquad \vec b'=\mat{5\\6}
				\]
				and $\vec a=\vec a'/\|\vec a'\|$ and $\vec b=\vec b'/\|\vec b'\|$
				and consider the basis $\mathcal B=\{\vec a,\vec b\}$.
			

			Use Matlab/Octave to compute the {\bf average} of $S_{\mathcal B}(\vec v)$
				for at least 1000 vectors with random entries in the interval $[-1,1]$.  (Hint:
				you can use \texttt{(rand(2,1) - .5)*2} to create a random vector with entries
				in $[-1,1]$).
			
			\item Let $\mathcal S=\{\hat {\mathbf x},\hat {\mathbf y}\}$
				be the standard basis for $\R^2$.
				Use Matlab/Octave to compute the average of $S_{\mathcal S}(\vec v)$
				for at least 1000 vectors with random entries in the interval $[-1,1]$.

			\item Consider the basis $\mathcal Q_{\theta}=\{\hat {\mathbf x}, \vec v_{\theta}\}$
					where $\vec v_\theta = \mat{\cos\theta\\\sin\theta}$  and
					$\theta \neq n\pi$.

					What happens to the average value of $S_{\mathcal Q_{\theta}}$
					as $\theta\to 0$?  Why?

			\item We're going to numerically explore the space of unit-vector bases
				for $\R^2$.  Let $\vec b_t  = \mat{\cos t\\\sin t}$ and let
				$\mathcal B_{\theta_1,\theta_2}=\left\{\vec b_{\theta_1},\vec b_{\theta_2}\right\}$
				where $\theta_1,\theta_2$ are randomly chosen from the interval $[0,2\pi]$.

				\begin{enumerate}
					\item What percentage of random bases for $\R^2$ satisfy $S_{\mathcal B} < 2$?
					\item What percentage satisfy $S_{\mathcal B} > 10$?  
					\item What is the smallest
				that $S_{\mathcal B}$ can be?  
					\item What do bases achieving minimal values
				for $S_{\mathcal B}$ look like?
				\end{enumerate}

			\item Now let's explore bases for $\R^3$.  For a basis of unit vectors, $\mathcal B$, in
				$\R^3$, we define $S_{\mathcal B}$ in the same way.  However, choosing a
				random basis of unit vectors is a little harder in $\R^3$.  (We cannot just
				use spherical coordinates and randomly pick $\phi$ and $\theta$ since that
				would cause most of the vectors to collect at the north and south poles---remember
				what the volume form looks like for polar coordinates?)  However, we can use a trick.

				The Matlab/Octave command \texttt{randn(3,1)} will create a random vector
				with three \emph{normally distributed} components.  If we turn this vector
				into a unit vector (something like \texttt{a = randn(3,1); a = a/norm(a)})
				it will be randomly (and uniformly) distributed on the unit sphere in $\R^3$.

				Find out what percentage of random bases for $\R^3$ satisfy $S_{\mathcal B} < 2$.
				Provide a histogram of $S_{\mathcal B}$ for random bases for $\R^3$.  (The
				Matlab/Octave command \texttt{hist(v)} with give you a histogram if \texttt{v}
				is a list of numbers.)
			
		\end{enumerate}



	\end{enumerate}


\end{document}
