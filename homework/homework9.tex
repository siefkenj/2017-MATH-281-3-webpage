\documentclass[letter]{article}
\usepackage{amsmath}
\usepackage{amsfonts}
\usepackage{amssymb}
\usepackage{ifthen}
\usepackage{fancyhdr}
\usepackage[usenames,dvipsnames,svgnames,table]{xcolor}
\usepackage{tikz}

%%%
% Set up the margins to use a fairly large area of the page
%%%
\oddsidemargin=.2in
\evensidemargin=.2in
\textwidth=6in
\topmargin=0in
\textheight=9.0in
\parskip=.07in
\parindent=0in
\pagestyle{fancy}

\expandafter\def\expandafter\quote\expandafter{\quote\sf\color{DarkGreen}}

%%%
% Set up the header
%%%
\newcommand{\setheader}[6]{
	\lhead{{\sc #1}\\{\sc #2} %({\small \it \today})
	}
	\rhead{
		{\bf #3} 
		\ifthenelse{\equal{#4}{}}{}{(#4)}\\
		{\bf #5} 
		\ifthenelse{\equal{#6}{}}{}{(#6)}%
	}
}

%%%
% Set up some shortcut commands
%%%
\newcommand{\R}{\mathbb{R}}
\newcommand{\N}{\mathbb{N}}
\newcommand{\Z}{\mathbb{Z}}
\newcommand{\Proj}{\mathrm{proj}}
\newcommand{\Perp}{\mathrm{perp}}
\newcommand{\Span}{\mathrm{span}}
\newcommand{\Null}{\mathrm{null}}
\newcommand{\Rank}{\mathrm{rank}}
\newcommand{\mat}[1]{\begin{bmatrix}#1\end{bmatrix}}

%%%
% This is where the body of the document goes
%%%
\begin{document}
	\setheader{Math 281-3}{Homework 9}{Due Monday, June 6}{}{}{}

		
			We've come a long way in this course---we've gone from row-reduction to matrix equations, from
			spans and linear independence to eigenvectors and diagonalization.  I would like you to
			write reflect about your experience.

			Look back on the three over-arching {\sc Learning Outcomes} listed on the course syllabus.
			Imagine you are writing a letter to a parent/aunt/uncle who has 
			taken technical math/science courses in
			the past, but has forgotten most of the terminology.  Write to this audience explaining
			to what level you feel you've achieved (or not achieved) each outcome.  Your essay
			may talk about Math 281-3 exclusively, or you may talk about the entire Math 281 sequence.
			As you write, try to give specific examples/anecdotes to back up your claims (for example, 
			``I didn't see the utility in having multiple definitions for a single object until
			I had to visualize linear independence and then prove something about it.  Visualizing, it was
			much easier to think in terms of the \emph{no vector can be removed} definition, but
			when proving things, the \emph{no non-trivial linear combination gives the zero vector}
			definition was easier to handle\mbox{.}'' is much better than ``I didn't see the utility in 
			having multiple definitions for a single object until the homework.'')


			Your essay should be 1--2 pages in length.  To help you set the mood, 
			consider writing your essay as a letter, perhaps starting with,
			\begin{quote}
				Dear Mom, 

				This year at Northwestern, I've been taking Math 281, which is a course about$\ldots$
			\end{quote}

			Of course, if you're inspired to start your essay a different way, feel free to!

			Some guidelines on how your essay will be judged:
			\begin{enumerate}
				\item[Acceptable:]
					Correctly identified the learning outcomes from the syllabus 
					and conveyed to the reader whether or not you have achieved them.
				\item[Good:]
					Correctly identified the learning outcomes from the syllabus
					and explained clearly \emph{why} you have or have not achieved them.
				\item[Excellent:]
					Correctly identified the learning outcome from the syllabus, 
					discussed \emph{how} these learning outcomes connect to specific material or themes from the course, 
					and explained clearly \emph{why} you have or have not achieved them.
			\end{enumerate}

\end{document}
